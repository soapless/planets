\doublespace
\newpage
%%%%%%original format %%%%%%%%
%\date{}
\section*{\uppercase{Supplementary Document}}
\setcounter{page}{1}
\setcounter{figure}{0}
{\large For: Understanding the Impact of Stroke on Brain Motor Function: A Hierarchical Bayesian Approach}\\
Author: Zhe Yu, Raquel Prado, Erin Burke Quinlan, Steven C. Cramer, and Hernando Ombao \\
Date: \today
%\date{}
\subsection*{Full Conditional Posterior Distributions}
Define notation $v(i_1:i_2)$ to be the subset of vector $v$ from elements $i_1$ to $i_2$. Subscript ``$post$'' is used to indicate the quantity is based on posterior distribution. Without loss of {\color{blue}generality}, we only need to list the posterior distributions for a single group. For notational convenience, we drop all the superscripts for group. 
Subject level:
\setcounter{equation}{0}
\begin{eqnarray}
\label{post-beta}
\vbeta\s &\sim& N(\vmu_{\vbeta,post}\s, \Sigma_{\vbeta,post}\s)\\
\label{post-phi}
\vphi\s &\sim& N(\vphi_{\vphi,post}\s, \Sigma_{\vphi,post}\s)\\
\label{post-d}
\vd\s &\sim& N(\vmu_{\vd,post}\s, \Sigma_{\vd,post}\s)\\
\label{post-xi}
\vxi_{pq}\s &\sim& Multi{\color{blue}nomial}(1, \vpi_{pq, post}\s)\\
\label{post-omega}
\Omega\s &\sim& Wishart(\Omega_{post}\s, \nu_{\Omega,post}\s) 
\end{eqnarray}
where   
\begin{eqnarray}
\Sigma_{\vbeta,post}\s&=& (R\sum_{t=L+1}^{T}(\tilde{X}\s(t))'\Omega\s\tilde{X}\s(t)+{\color{blue}(\Sigma_{\vbeta}\gs)}^{-1})^{-1} \\
\vmu_{\vbeta,post}\s &=& \Sigma_{\vbeta,post}\s(\sum_{r=1}^{R}\sum_{t=L+1}^{T}(\tilde{X}\s(t))'\Omega\s\tilde{\vy}\sr(t)+{\color{blue}(\Sigma_{\vbeta}\gs)^{-1}\vmu_{\vbeta}\gs})\\
\Sigma_{\vd,post}\s&=& (R\sum_{t=L+1}^{T}(\tilde{Z}\s(t))'\Omega\s\tilde{Z}\s(t)+{\color{blue}(\Sigma_{\vd}\gs)}^{-1})^{-1} \\
\vmu_{\vd,post}\s &=& \Sigma_{\vd,post}\s(\sum_{r=1}^{R}\sum_{t=L+1}^{T}(\tilde{Z}\s(t))'\Omega\s\tilde{\vy}_0\sr(t)+{\color{blue}(\Sigma_{\vd}\gs)^{-1} \vmu_{\vd}\gs})\\
\Sigma_{\vphi,post}\s&=& ({\color{blue}\sum_{r=1}^R}\sum_{t=L+1}^{T}(\tilde{W}\sr(t))'\Omega\s\tilde{W}\sr(t)+ {\color{blue}(\Sigma_{\vphi}\gs)}^{-1})^{-1} \\
\vmu_{\vphi,post}\s &=& \Sigma_{\vphi,post}\s(\sum_{r=1}^{R}\sum_{t=L+1}^{T}(\tilde{W}\sr(t))'\Omega\s\vu\sr(t)+ {\color{blue}(\Sigma_{\vphi}\gs)^{-1} \vmu_{\vphi}\gs})\\
%\end{eqnarray}
%\begin{eqnarray}
%\vpi_{pq, post}\s &=& [\vpi_{pq, post}\s(0), \dots, \vpi_{pq, post}\s(L)] \\
%\vpi_{pq, post}\s(j) &\propto& \vpi_{pq}\s(j)^{I(\vxi_{pq}\s=\vell_j)}\lik_{\vxi_{pq}\s}(\vell_j), ~j=0,\dots, L; ~\text{and } \sum_{j=0}^L \vpi_{pq, post}\s(j) = 1 \\%\frac{}{\sum_{j=0}^L \vpi_{pq, post}\s(j)} \\
\vpi_{pq, post}\s(j) &\propto& {\color{blue}\vpi_{pq}\gs(j)}^{I(\vxi_{pq}\s=\vell_j)}\times \exp\{ -\half\sum_{r=1}^{R}\sum_{t=L+1}^{T}(\vu\sr(t)-W\sr(t)\vphi\s_{(pq,j)})' \\
&\times&\Omega\s(\vu\sr(t)-W\sr(t)\vphi\s_{(pq,j)})\}\\
j&=&0,\dots, L; ~\text{and } \sum_{j=0}^L \vpi_{pq, post}\s(j) = 1 \\%\frac{}{\sum_{j=0}^L \vpi_{pq, post}\s(j)} \\
\Omega_{post}\s&=& (SSE\s + {\color{blue}(\Omega\gs)}^{-1})^{-1}\\
\nu_{\Omega,post}\s&=& R\times(T - L) +\nu_{\Omega}\\
\label{SSE}
SSE\s &=& \sum_{r=1}^{R}\sum_{t=L+1}^{T}(\tilde{\vy}\sr(t) - \tilde{X}\s(t)\vbeta\s)(\tilde{\vy}\sr(t) - \tilde{X}\s(t)\vbeta\s)' 
\end{eqnarray}

Group level:
\begin{eqnarray}
\label{beta-group}
{\color{blue}\vmu_{\vbeta}^g} &\sim& N(\vmu_{\vmu_{\vbeta},post},\Sigma_{\vmu_{\vbeta},post})\\
\label{phi-group}
{\color{blue}\vmu_{\vphi}^g} &\sim& N(\vmu_{\vmu_{\vphi},post},\Sigma_{\vmu_{\vphi},post})\\
\label{d-group}
{\color{blue}\vmu_{\vd}^g} &\sim& N(\vmu_{\vmu_{\vd},post},\Sigma_{\vmu_{\vd},post})\\
\label{sigbeta-group}
{\color{blue}(\Sigma_{\vbeta}^g(i,i))}^{-1} &\sim& Gamma(a_{\vbeta_i,post},b_{\vbeta_i,post}), ~ i =1, \dots, P\\
\label{sigphi-group}
{\color{blue}(\Sigma_{\vphi}^g(i,i))}^{-1} &\sim& Gamma(a_{\vphi_i,post},b_{\vphi_i,post}),~ i =1, \dots, 2LP^2\\
{\color{blue}(\Sigma_{\vd}^g(i,i))}^{-1} &\sim& Gamma(a_{\vd_i,post},b_{\vd_i,post}),~ i =1, \dots, PJ\\
\label{pi-group}
{\color{blue}\vpi_{pq}^g} &\sim& Dir(\valpha_{\vpi,post})\\
\label{omega-group}
{\color{blue}(\Omega^{g})}^{-1} &\sim& Wishart(\Sigma_{post}, \nu_{post}) 
\end{eqnarray}
where
\begin{eqnarray}
\vmu_{\vmu_{\vbeta},post}&=& \Sigma_{\vmu_{\vbeta},post}(\Sigma_{\vbeta}^{-1}\sum_{s=1}^S \vbeta\s + \Sigma_{\vmu_{\vbeta},0}^{-1}\vmu_{\vmu_{\vbeta},0})\\
\Sigma_{\vmu_{\vbeta},post}&=& (S\Sigma_{\vbeta}^{-1} + \Sigma_{\vmu_{\vbeta},0}^{-1})^{-1}\\
\vmu_{\vmu_{\vphi},post}&=& \Sigma_{\vmu_{\vphi},post}(\Sigma_{\vphi}^{-1}\sum_{s=1}^S \vphi\s + \Sigma_{\vmu_{\vphi},0}^{-1}\vmu_{\vmu_{\vphi},0})\\
\Sigma_{\vmu_{\vphi},post}&=& (S\Sigma_{\vphi}^{-1} + \Sigma_{\vmu_{\vphi},0}^{-1})^{-1}\\
\vmu_{\vmu_{\vd},post}&=& \Sigma_{\vmu_{\vd},post}(\Sigma_{\vd}^{-1}\sum_{s=1}^S \vd\s + \Sigma_{\vmu_{\vd},0}^{-1}\vmu_{\vmu_{\vd},0})\\
\Sigma_{\vmu_{\vd},post}&=& (S\Sigma_{\vd}^{-1} + \Sigma_{\vmu_{\vd},0}^{-1})^{-1}\\
a_{\vbeta_i,post}&=&\half S + a_{\vbeta}, ~~
b_{\vbeta_i,post}= \half\sum_{s=1}^S (\vbeta\s_p)^2+ b_{\vbeta}\\
a_{\vphi_i,post}&=&\half S + a_{\vphi}, ~~
b_{\vphi_i,post}=\half\sum_{s=1}^S (\vphi\s_p)^2 + b_{\vphi}\\
%a_{\vd_i,post}&=&\half S + a_{\vd}\\
%b_{\vd_i,post}&=&\half\sum_{s=1}^S (\vd\s_p)^2 + b_{\vd}\\
\valpha_{\vpi,post}(j)&=& \sum_{s=1}^S I(\vxi_{pq}\s=\vell(j)) + \valpha_{\vpi}\\
\Sigma_{post} &=& (\sum_{s=1}^S \Omega\s + \Sigma_0^{-1})^{-1}, ~~
\nu_{post} = S\nu_{\Omega} + \nu_0
\end{eqnarray}


%\section*{\uppercase{Appendix}-ii}
\subsection*{Sampling Algorithm}
\begin{enumerate}
  \item \label{sample-initialize}
Initialize:
	\begin{enumerate}  
	\item Set the initial values for group-level parameters: $\Sigma_{\vphi}(i,i)$ and $\Sigma_{\vbeta}(i,i)$ set to relatively large values, $\Omega^{g_s} = I$. 
	{\color{blue}
	\item Initialize the subject-level parameters: calculate the posterior modes for $\vbeta\s$ and $\vd\s$ by conditional maximization, based on the fixed group parameters and $\Omega\s=I$, $\vphi\s=\vzero$; initialize $\vbeta\s$ and $\vd\s$ by those posterior modes. Apply VAR($L$) on the resulting residuals and calculate the point estimates $\hat{\vphi}\s$ and $\hat{\Omega}\s$; initialize $\vphi\s$ and $\xi_{pq}\s(\ell)$ as $[\hat{\vphi}\s, \hat{\vphi}\s]$ and $\vell_j$ s.t. $j = argmax_{0\le j'\le L}\{\hat{\phi}_{pq}\s(\ell)<0.1, ~\forall \ell=1,\dots, j'\}$, and initialize $\Omega\s$ as $\hat{\Omega}$. 
%from Equation \eqref{h_conv_s} in the paper separately for each subject, and assign to be the intial values of $\vbeta\s$ and $\vd\s$, respectively.	 
	}
%	\item Initialize subject level parameters by sampling $\vphi_{i}\s \stackrel{ind}{\sim} N(0, \frac{3}{10}), ~ i=1,2,\dots, 2LP^2$, %$\vd\s \sim N(\hat{\vmu}_{\vd}, \hat{\Sigma}_{\vd})$, $\vxi_{pq}\s =\vell(0), ~p,q=1,\dots, P$, $\Omega\s = (\frac{1}{R(T - L)}\sum_{t,r} (\vy\sr(t)-\bar{\vy}\s)(\vy\sr(t)-\bar{\vy}\s)')^{-1}$, where $\bar{\vy}\s = \frac{1}{T\cdot R}\sum_{t,r} \vy\sr(t)$. Calculate the OLSE from Equation \eqref{h_conv_s} in the paper separately for each subject, and assign to be the intial values of $\vbeta\s$ and $\vd\s$, respectively.
	\end{enumerate}  
  \item \label{sample-subject}
Sample {\color{blue}the subject parameters} for each subject $s$ separately: based on the latest value of all other parameters, 
	\begin{enumerate}  
	  \item Sample from posterior \eqref{post-phi}, \eqref{post-xi}. Update $\vphi\sst$, $\tilde{\vy}\sr(t)$ and $\tilde{Z}\s(t)$ according to Equation \eqref{phistar}, \eqref{ytilde}, and \eqref{Ztilde} in the paper.
	  \item Sample from posterior \eqref{post-d}. Update $\tilde{X}\s(t)$ according to Equation \eqref{Xtilde} in the paper.
	   \item Sample from posterior \eqref{post-beta}. Update $\tilde{\vy}_0\sr(t)$, $U\sr(t)$, $\tilde{W}\sr(t)$ and $SSE\s$ according to Equation \eqref{y0tilde}, \eqref{Wtilde1}, and \eqref{Wtilde3} in the paper, and Equation \eqref{SSE}.
	   \item Sample from posterior \eqref{post-omega}. 
	\end{enumerate}  
 \item \label{sample-pilot}
Refine the initial values for the subject parameters by repeating only Step \ref{sample-subject} for a certain amount of iterations (pilot estimation), and continue {\color{blue}to} the next steps (formal estimation) {\color{blue}with} the latest iteration. 
  \item \label{sample-group}
Aggregate the latest information across subjects and update for group level: 
%  	\begin{enumerate}  
%     \item 
Sample from posterior \eqref{beta-group}, \eqref{phi-group}, %\eqref{d-group}, 
     \eqref{sigbeta-group}, \eqref{sigphi-group}, \eqref{phi-group}, \eqref{omega-group}.
%	\end{enumerate}  
 \item \label{sample-update}
Repeat Step \ref{sample-subject} for all the subjects for 5 iterations, and then repeat Step \ref{sample-group}.
 \item \label{sample-repeat}
Repeat the previous step until the chain reaches the pre-specified size (e.g. 5,000) after a certain burn-in period of Step \ref{sample-update}.   
\end{enumerate}




%\section*{\uppercase{Appendix}-iii}
\subsection*{Parameter Values Used in Simulation Study} 
\begin{itemize}
\item Mean and covariance matrix for generating $\vbeta\s$:
\vspace{-.1in}
\begin{eqnarray*}
\E\vbeta &=& [27,  3,  18, 30, 33; 15,  3, 23, 4, 26; \vzero_5']'\\
%\end{eqnarray}
%\begin{eqnarray}
\Cov(\beta_{pk}, \beta_{qj}) &=&
\begin{cases}
3.556, &\text{ if } p=q, ~p\ne 2, ~k=j\\
%0.395, &\text{ if } p=q, ~p\ne 2, ~k\ne j
0.5000, &\text{ if } p=q=2, ~k=j\\
%0.0555, &\text{ if } p=q=2, ~k\ne j
\frac{1}{9}\sqrt{\Cov(\beta_{pk}, \beta_{pk})\Cov(\beta_{pj}, \beta_{pj}}), &\text{ if } p=q, ~k\ne j\\
0, &\text{ if } p\ne q
\end{cases}
\end{eqnarray*}

\item Covariance matrix for generating $\vphi\s$: $0.05^2I_{2LP^2}$
\item HRF basis coefficient $\vd\s$:
\begin{eqnarray*}
\E\vd_1\s &=& [1.9783, ~ 2.4983, ~ 1.7715,~ -0.7298, ~ 0.3259]\\
\E\vd_2\s &=& [2.5797, ~-0.2253,~  0.1054, ~ 0.5102,~ -0.1120]\\
\E\vd_3\s = \E\vd_4\s &=& [2.4341, ~ 0.4343, ~ 0.5088, ~ 0.2099,~ -0.0059]\\
\E\vd_5\s &=& [2.7224,~ -0.8712,~ -0.2898, ~ 0.8043,~ -0.2158]\\ 
\Cov(\vd_{p}) &=& diag\{0.05587,~ 0.2529, ~0.1547, ~0.1152, ~0.04066\}\\
\vd_{p} &\perp& \vd_{q}, ~ p\ne q
\end{eqnarray*}

% > round(t(dmat_group),4)
         % H1      H2      H3      H4      H5
% [1,] 1.9783  2.4983  1.7715 -0.7298  0.3259
% [2,] 2.5797 -0.2253  0.1054  0.5102 -0.1120
% [3,] 2.4341  0.4343  0.5088  0.2099 -0.0059
% [4,] 2.4341  0.4343  0.5088  0.2099 -0.0059
% [5,] 2.7224 -0.8712 -0.2898  0.8043 -0.2158

\item Covariance matrix of $\veps$ (the same across subjects)
% > Sigma
           % [,1]         [,2] [,3] [,4]         [,5]
% [1,] 16.0112579  0.300211086    0    0 -0.300211086
% [2,]  0.3002111 16.005628958    0    0 -0.005628958
% [3,]  0.0000000  0.000000000   16    0  0.000000000
% [4,]  0.0000000  0.000000000    0   16  0.000000000
% [5,] -0.3002111 -0.005628958    0    0 16.005628958
\begin{eqnarray*}
(\Sigma\s)^{-1} = \Omega\s = 
\begin{bmatrix}
0.0625 & -0.00117 & 0 & 0 & 0.00117 \\
-0.00117 & 0.0625 & 0 & 0 & 0 \\
0 & 0 & 0.0625 & 0 & 0 \\
0 & 0 & 0 & 0.0625 & 0 \\
0.00117 & 0 & 0 & 0 & 0.0625 
\end{bmatrix}
\end{eqnarray*}

\end{itemize}

{\color{blue}
\subsection*{VB Algorithm}
}
% \begin{eqnarray}
% \label{VB-beta}
% \vbeta\s &\sim& N(\vmu_{\vbeta,VB}\s, \Sigma_{\vbeta,VB}\s)\\
% \label{VB-phi}
% \vphi\s &\sim& N(\vphi_{\vphi,VB}\s, \Sigma_{\vphi,VB}\s)\\
% \label{VB-d}
% \vd\s &\sim& N(\vmu_{\vd,VB}\s, \Sigma_{\vd,VB}\s)\\
% \label{VB-xi}
% \vxi_{pq}\s &\sim& Multi(1, \vpi_{pq, VB}\s)\\
% \label{VB-omega}
% \Omega\s &\sim& Wishart(\Omega_{VB}\s, \nu_{\Omega,VB}\s) 
% \end{eqnarray}
% where 
We first introduce some notations. Let $\odot$ denote element-wise matrix or vector multiplication, and $Bdiag_2\{\vx\}$ denote $Bdiag\{\vx, \vx\}$. 
$\tilde{\mu}_{\Phi,k}\s(\ell)$ denotes the matrix formed by the corresponding elements in the vector $\tilde{\mu}_{\vphi}\s$ for lag $\ell$ and condition $k$; 
$\tilde{\mu}_{\xi_{pq}\s(\ell)}$ denotes $\E_q\xi_{pq}\s(\ell)$; 
$\tilde{\mu}_{\vxi}\s(\ell)$ denotes the matrix $[\tilde{\mu}_{\xi_{pq}\s(\ell)}]_{p,q=1}^{P}$; 
$\tilde{\Sigma}_{\vxi_{pq}\s(\ell), \vxi_{p'q'}(\ell')}\s$ denotes $\Cov_q(\xi_{pq}\s(\ell), \xi_{p'q'}\s(\ell'))$; 
%denotes $[\vecn\{\tilde{\mu}_{\vxi}\s(1)\}', \dots, \vecn\{\tilde{\mu}_{\vxi}\s(L)\}']'$ 
$\tilde{\mu}_{\vxi}\s$ and $\tilde{\Sigma}_{\vxi}\s$ denotes the mean and the covariance matrix under $q(\cdot)$ for $\vxi\s=[\vecn\{{\vxi}\s(1)\}', \dots, \vecn\{{\vxi}\s(L)\}']'$ and they are formed by the elements of $\tilde{\mu}_{\vxi}\s(\ell)$ and $\tilde{\Sigma}_{\vxi_{pq}\s(\ell); 
\vxi_{p'q'}(\ell')}\s$, $\tilde{\mu}_{\beta,pk}\s$ denotes the element in $\tilde{\mu}_{\vbeta}\s$ for ROI $p$ and condition $k$; 
$\tilde{\mu}_{\vbeta,0}\s$ denotes the element in $\tilde{\mu}_{\vbeta}\s$ for all the intercepts; 
$\tilde{\mu}_{\vd,p}\s$ denotes the elements in $\tilde{\mu}_{\vd}\s$ for ROI $p$; 
$\vi_{\ell}$ denotes the index vector $[1,\dots,\ell, L+1, \dots, L+\ell]'$; 
$\vi^{\phi}_{pq}$ denotes the index vector $[p+(q-1)P+(1-1)P^2, \dots, p+(q-1)P+(L-1)P^2, p+(q-1)P+(1-1)P^2+ LP^2, \dots, p+(q-1)P+(L-1)P^2 + LP^2]'$ and $-\vi^{\phi}_{pq}$ denotes the complement of $\vi^{\phi}_{pq}$ with respective to $1:(2LP^2)$; 
$\vi^{\xi}_{pq}$ denotes the index vector $[p+(q-1)P+(1-1)P^2, \dots, p+(q-1)P+(L-1)P^2]'$ and $-\vi^{\xi}_{pq}$ denotes the complement of $\vi^{\xi}_{pq}$ with respective to $1:(LP^2)$; 
$\vx(\vi)$ denotes the vector formed by $\vx$'s elements ($\vi=[\vi(1),\vi(2),\dots, \vi(n)]$); 
$A(:,\vi)$ (or $A(\vi,:)$) denotes the matrix formed by $A$'s $\vi(1)$-th, $\vi(2)$-th, $\dots$, and $\vi(n)$-th columns (or rows); 
$\tilde{\mu}_{\vxi,{pq}}$ denotes $\tilde{\mu}_{\vxi}(\vi^{\xi}_{pq})$, and $\tilde{\mu}_{\vxi,{-pq}}$ denotes $\tilde{\mu}_{\vxi}(-\vi^{\xi}_{pq})$. 

The VB algorithm iteratively updates $\tilde{\mu}_{\vbeta}\s$, $\tilde{\Sigma}_{\vbeta}\s$, $\tilde{\mu}_{\vd}\s$, $\tilde{\Sigma}_{\vd}\s$, $\tilde{\mu}_{\vphi}\s$, $\tilde{\Sigma}_{\vphi}\s$, $\tilde{\vpi}_{pq}\s$, $\tilde{\nu}_{\Omega}\s$ and $\tilde{\mu}_{\Omega}\s$ until convergence, according to the following equations: 
\begin{eqnarray*}
\tilde{\vPhi}_t\s(\text{B}) &=& I_P - \sum_{\ell=1}^L\sum_{k=1}^2 c_k(t-\ell)\tilde{\mu}_{\Phi,k}\s(\ell)\odot \tilde{\mu}_{\vxi}\s(\ell)\text{B}^{\ell}\\ 
%Z\s(t)&=& Bdiag\{\sum_{k=1}^2\Lambda_k(t,:)\mu_{\beta,1k}\s, \dots, \sum_{k=1}^2\Lambda_k(t,:)\mu_{\beta,Pk}\s\}\\
\tilde{Z}\s(t)&=& \tilde{\vPhi}_t\s(\text{B})[Bdiag\{\sum_{k=1}^2\Lambda_k(t,:)\tilde{\mu}_{\beta,1k}\s, \dots, \sum_{k=1}^2\Lambda_k(t,:)\tilde{\mu}_{\beta,Pk}\s\}]\\
\tilde{\vy}_0\sr(t) & = & \tilde{\vPhi}_t\s(\text{B})[\vy\sr(t) - \tilde{\mu}_{\vbeta,0}\s]\\
\tilde{\Sigma}_{\vd}\s&=& (R\sum_{t=L+1}^{T}\tilde{Z}\s(t)'\tilde{\mu}_{\Omega}\s\tilde{Z}\s(t)+(\Sigma_{\vd}\gs)^{-1})^{-1} \\
\tilde{\mu}_{\vd}\s &=& \tilde{\Sigma}_{\vd}\s(\sum_{r=1}^{R}\sum_{t=L+1}^{T}\tilde{Z}\s(t)'\tilde{\mu}_{\Omega}\s\tilde{\vy}_0\sr(t)+(\Sigma_{\vd}\gs)^{-1}\vmu_{\vd}\gs)\\
X\s(t)&=& Bdiag\{1,~\Lambda_1(t,:)\tilde{\mu}_{\vd,1}\s, ~\Lambda_2(t,:)\tilde{\mu}_{\vd,1}\s, \dots, 1,~\Lambda_1(t,:)\tilde{\mu}_{\vd,P}\s, ~\Lambda_2(t,:)\tilde{\mu}_{\vd,P}\s\}\\
\tilde{X}\s(t)&=& \tilde{\vPhi}_t\s(\text{B})[X\s(t)]\\
\tilde{\vy}\sr(t) & = & \tilde{\vPhi}_t\s(\text{B})[\vy\sr(t)]\\
\tilde{\Sigma}_{\vbeta}\s &=& (R\sum_{t=L+1}^{T}\tilde{X}\s(t)'\tilde{\mu}_{\Omega}\s\tilde{X}\s(t)+(\Sigma_{\vbeta}\gs)^{-1})^{-1} \\
\tilde{\mu}_{\vbeta}\s &=& \tilde{\Sigma}_{\vbeta}\s(\sum_{r=1}^{R}\sum_{t=L+1}^{T}\tilde{X}\s(t)'\tilde{\mu}_{\Omega}\s\tilde{\vy}\sr(t)+(\Sigma_{\vbeta}\gs)^{-1}\mu_{\vbeta}\gs)\\
\tilde{\vu}\sr(t)&=& \vy\sr(t) - X\s(t)\tilde{\mu}_{\vbeta}\s\\
\tilde{W}_t\sr &=& [c_1(t-1)\tilde{\vu}\sr(t)'\otimes I_P, \dots,  c_1(t-L)\tilde{\vu}\sr(t)'\otimes I_P, \\
&& c_2(t-1)\tilde{\vu}\sr(t)'\otimes I_P, \dots,  c_2(t-L)\tilde{\vu}\sr(t)'\otimes I_P]\\
\tilde{\Sigma}_{\vphi}\s&=& (\sum_{r=1}^R\sum_{t=L+1}^{T}(\tilde{W}_t\sr)'\tilde{\mu}_{\Omega}\s\tilde{W}_t\sr\odot Bdiag_2(\tilde{\mu}_{\vxi}\s(\tilde{\mu}_{\vxi}\s)' + \tilde{\Sigma}_{\vxi}\s) +(\Sigma_{\vphi}\gs)^{-1})^{-1} \\
\tilde{\mu}_{\vphi}\s &=& \tilde{\Sigma}_{\vphi}\s(diag\{\tilde{\mu}_{\vxi}\s,\tilde{\mu}_{\vxi}\s\}\sum_{r=1}^{R}\sum_{t=L+1}^{T}(\tilde{W}_t\sr)'\tilde{\mu}_{\Omega}\s\tilde{\vu}\sr(t)+(\Sigma_{\vphi}\gs)^{-1}\mu_{\vphi}\gs)\\
\nonumber
&&\text{Loop over $(p,q)\in \{1,2,\dots, P\}^2$ in a random order:}\\
A_1 &=& \sum_{r=1}^{R}\sum_{t=L+1}^{T}\tilde{W}_t\sr(:,\vi^{\phi}_{pq})'\tilde{\mu}_{\Omega}\s\tilde{W}_t\sr(:,\vi^{\phi}_{pq})\left(\tilde{\mu}_{\vphi}\s(\vi^{\phi}_{pq})\tilde{\mu}_{\vphi}\s(\vi^{\phi}_{pq})' + \tilde{\Sigma}_{\vphi}\s(\vi^{\phi}_{pq}, \vi^{\phi}_{pq}) \right)\\
A_2 &=& \sum_{r=1}^{R}\sum_{t=L+1}^{T}[\tilde{W}_t\sr(:, \vi^{\phi}_{pq})'\tilde{\mu}_{\Omega}\s\tilde{W}_t\sr(:, -\vi^{\phi}_{pq}) diag\{\tilde{\mu}_{\vxi,-pq}\s, \tilde{\mu}_{\vxi,-pq}\s\} \\
&&\times (\tilde{\mu}_{\vphi}\s(-\vi^{\phi}_{pq})'\tilde{\mu}_{\vphi}\s(\vi^{\phi}_{pq})
+\tilde{\Sigma}_{\vphi}\s(-\vi^{\phi}_{pq}, \vi^{\phi}_{pq}))\\
A_3 &=& \sum_{r=1}^{R}\sum_{t=L+1}^{T}\tilde{W}_t\sr(:, \vi^{\phi}_{pq})'\tilde{\mu}_{\Omega}\s \tilde{\vu}\sr(t)\\
\tilde{\tilde{\vpi}}_{pq}\s(\ell)&= & log(\vpi_{pq}\s(\ell)\gs) -\frac{1}{2}[ tr(A_1[\vi_{\ell},\vi_{\ell}]) + 2tr(A_2[\vi_{\ell},\vi_{\ell}]) - 2\tilde{\mu}_{\phi}\s(\vi_{\ell})'A_3(\vi_{\ell})]\\
\ell&=&1,2,\dots, L\\
\tilde{\vpi}_{pq}\s(\ell) &=& \tilde{\tilde{\vpi}}_{pq}\s(\ell)/\sum_{j=0}^L \tilde{\tilde{\vpi}}_{pq}\s(j)\\ 
\mu_{\vxi,pq}\s(\ell) &=& \sum_{j=1}^{\ell}\tilde{\vpi}_{pq}\s(j)\\
\tilde{\Sigma}_{\vxi_{pq}(\ell), \vxi_{pq}(\ell')}\s &=& \sum_{j=1}^{\min\{\ell,\ell'\}}\mu_{\vxi, pq}\s(j) - \sum_{j=1}^{\ell}\sum_{j'=1}^{\ell'}\mu_{\xi,pq}\s(j)\mu_{\xi,pq}\s(j')\\ 
\tilde{\Sigma}_{\vxi_{pq}(\ell),\vxi_{p'q'}(\ell')}\s &=& 0, ~\forall (p,q)\ne(p',q')\\
\nonumber 
&&\text{End of loop.}\\
\label{SSE}
SSE\s &=& \sum_{r=1}^{R}\sum_{t=L+1}^{T}[\tilde{\vu}\sr(t)\tilde{\vu}\sr(t)' - \tilde{W}_t\sr\tilde{\mu}_{\vphi}\odot [\tilde{\mu}_{\vxi}', \tilde{\mu}_{\vxi}']'\tilde{\vu}\sr(t)' \\
&&- (\tilde{W}_t\sr\tilde{\mu}_{\vphi}\odot [\tilde{\mu}_{\vxi}', \tilde{\mu}_{\vxi}']'\tilde{\vu}\sr(t)')'\\
&& + \tilde{W}_t\sr Bidag_2(\tilde{\mu}_{\vxi}\tilde{\mu}_{\vxi}'+\tilde{\Sigma}_{\vxi})(\tilde{\mu}_{\vphi}\tilde{\mu}_{\vphi}'+\tilde{\Sigma}_{\vphi})(\tilde{W}_t\sr)']\\ 
\tilde{\nu}_{\Omega}\s&=& R\times(T - L) +\nu_{\Omega} ~~\text{(constant)}\\
\tilde{\mu}_{\Omega}\s&=& \tilde{\nu}_{\Omega}\s(SSE\s + (\Omega\gs)^{-1})^{-1}
\end{eqnarray*}
Finally, we obtain 
\begin{eqnarray}
\vbeta\s &\sim& N(\tilde{\mu}_{\vbeta}, \tilde{\Sigma}_{\vbeta}), \\
\vd\s &\sim& N(\tilde{\mu}_{\vd}, \tilde{\Sigma}_{\vd}), \\
\vphi\s &\sim& N(\tilde{\mu}_{\vphi}, \tilde{\Sigma}_{\vphi}), \\
\vxi_{p,q}\s& \sim & Multinomial(1, \tilde{\vpi}_{pq}),\\
\Omega\s &\sim & Wishart(\tilde{\nu}_{\Omega}\s, ~\tilde{\mu}_{\Omega}\s/\tilde{\nu}_{\Omega}\s).
\end{eqnarray}

% \section*{\uppercase{Appendix}-iv}
% \subsection*{Additional Plots for Simulation Study} 
% \begin{itemize}
% \item example trace plots
% \item example ACF plots
% \end{itemize}

%%\section*{\uppercase{Appendix}-iv}
%\subsection*{Additional Plot for fMRI Data Analysis} 
%\begin{figure}[H]
%\centering
% \begin{minipage}[]{0.45\textwidth}
%      \centering
%\includegraphics[scale=.40]{./graph/realdata/res-fit.pdf}
%\end{minipage}
% \begin{minipage}[]{0.45\textwidth}
%      \centering
%\includegraphics[scale=.40]{./graph/realdata/res-fit2.pdf}
%\end{minipage}
%\caption[1]{Examples of mean fit (from a stroke patient) for LM1, LPMd, RM1, RPMd. Black curves are the actual time series, grey curves are posterior samples of mean structure, green lines separate different trials, blue lines indicate the task condition is on.}
%\label{res-fit}
%\end{figure}

%\newpage
